% This syllabus template was created by:
% Brian R. Hall
% Assistant Professor, Champlain College
% www.brianrhall.net

% Document settings
\documentclass[11pt]{article}
\usepackage[margin=1in]{geometry}
\usepackage[spanish]{babel}
\selectlanguage{spanish}
\usepackage[utf8]{inputenc}
\usepackage[pdftex]{graphicx}
\usepackage{multirow}
\usepackage{verbatim}
\usepackage{setspace}
\pagestyle{plain}
\setlength\parindent{0pt}
\usepackage{url}
\usepackage[colorlinks = true,
            linkcolor = blue,
            urlcolor  = blue,
            citecolor = blue,
            anchorcolor = blue]{hyperref}




\begin{document}

% Course information
\begin{tabular}{ l l }
  \multirow{2}{*}{\includegraphics[height=.75in,width=1.2in]{utdt.png}}  
  \\
 & \textbf{\large Maestría en Ciencia Política} 
  \\ 
  & Departamento de Ciencia Política y Estudios Internacionales
  
\end{tabular}
\vspace{20mm}

% Professor information
\begin{tabular}{l}
  \multirow{6}{*}  
\textbf{\LARGE Herramientas cuantitativas para el análisis político}
 \\
  \\

\large\textbf{Jueves, 19-22hs}
 \\

\large{(20 de septiembre al 29 de noviembre de 2018)}
 
  \\
  \\
 \large \textbf{Juan Pablo Pilorget}
 \\
 \large \href{mailto:jpilorget@gmail.com}{jpilorget@gmail.com}
 \\
  \\
 \large \textbf{Juan Pablo Ruiz Nicolini}
 \\
 \large \href{mailto:juan.ruiznicolini@mail.utdt.edu}{juan.ruiznicolini@mail.utdt.edu}
\end{tabular}

%\vspace{1cm}
    %\begin{center} \textit{\textbf{[Este es un programa tentativo sujeto a modificaciones]}} \\
%\end{center}

\vspace{2cm}

% Course details
\textbf {\large \\ Descripción del taller:} en su introducción a los conceptos básicos de las ciencias sociales, Jon Elster utilizó la metáfora de \textit{Tuercas y Tornillos}\footnote{\textit{Nuts and Bolts for the Social Science}, Cambridge University Press, 1989 / \textit{Tuercas y Tornillos:una introducción a los conceptos básicos de las ciencias sociales}; Gedisa, 1990.} para fundamentar la importancia de entender los engranajes detrás de los fenómenos sociales. Partiendo de esta idea, el taller pretende introducir un conjunto de  herramientas para la investigación y el análisis de fenómenos políticos que podrían ser pensados como los lubricantes de las \textit{tuercas y tornillos}. 

\vspace{.5cm}

Utilizaremos el entorno de \verb=RStudio= - para el manejo de datos, análisis estadístico e ilustración de resultados-, \verb=RMarkdown= para la presentación de documentos y una variedad de recursos de fácil acceso para mejorar la comunicación del trabajo académico. Tener conocimientos básicos de estadística y comprensión lectora en inglés será de gran ayuda para los participantes.

\vspace{.7cm}
Entre las motivaciones del taller se destacan:
\begin{itemize}
 \itemsep-0.4em
  \item  Introducir a los participantes en el descubrimiento y la utilización de herramientas de libre disposición y fácil acceso. 
  \item Desarrollar mediante la práctica una metodología de aproximación inicial a los fenómenos políticos a través del análisis exploratorio de datos y la inferencia estadística. 
\end{itemize}
\clearpage
\vspace{6mm}
\textbf {\large Lecturas}\\

Si bien a lo largo de las sesiones se compartirá material específico sobre los temas tratados, las siguientes lecturas fueron parte importante para el armado del programa: \\
\\
\emph{\textbf{*} \texttt{R} for Data Science}.
\textbf {Autores:} Hadley Wickham, Garrett Grolemund;  O' Reilly, 2016 \\ \\
\emph{\textbf{*} Political Analysis Using \texttt{R}}.
\textbf {Autor:} James E. Monogan III;  Springer, 2015\\ \\
\emph{\textbf{*} \textit{ggplot2: Elegant Graphics for Data Analysis}}.
\textbf {Autor:} Hadley Wickman;  Springer, 2009 \\
\\
\emph{\textbf{*}La introducción no tan corta a \LaTeX}.
\textbf {Autor:} Tobias Oetiker;  1995 \\

\emph{\textbf{*}Computer Age Statistical Inference}.
\textbf {Autores:} Bradley Efron, Trevor Hastie; Cambridge University Press, 2016 \\

\vspace{2cm}

\textbf {\large Objetivos del Taller} \\

Al completar el curso se espera que los participantes:
\begin{enumerate} \itemsep0.75em

  \item Conozcan y utilicen \verb=R= + \verb=RStudio= como \textit{software} para la manipulación de datos estructurados y la confección de modelos estadísticos.
  \item Sean capaces de producir gráficos y tablas de calidad en \verb=R= (utilizando \verb=ggplot2= y \verb=stargazer=, entre otras herramientas). 
  \item Usen \LaTeX y \verb=RMarkdown= para la creación y edición de documentos (artículos, presentaciones, bibliografía, etcétera).

  \item  Se familiaricen con herramientas adicionales para el trabajo con datos y la utilización de \textit{softwares} estadísticos (\verb=GitHub, StackOverflow, StackExchange=, etcétera). 
\end{enumerate}
\clearpage

% Course Outline
\textbf {\large Estructura del curso}:


\vspace{.5cm}
El contenido semanal podrá ser modificado en función del progreso de la clase. 

\vspace{.5cm}


\begin{table}[h!]
\normalsize % The size of the table text can be changed depending on content. Remove if desired.
\begin{tabular}{ | c | c | }
\hline
\textbf{Semana} & \textbf{Contenido} \\
\hline
Semana 1 & \begin{minipage}{.85\textwidth}
\begin{itemize} \itemsep-0.4em
	\vspace{1mm}
	\item \textbf{Lubricar las \textit{Tuercas y Tornillos}}\\
	Presentación del curso y trabajo preparatorio. 
	\vspace{1mm}
\end{itemize}
\end{minipage} \\

\hline\hline
Semana 2 & \begin{minipage}{.85\textwidth}
\begin{itemize} \itemsep-0.4em
	\vspace{1mm}
	\item \textbf{Introducción a \texttt{R}}.
	 
	\vspace{1mm}
\end{itemize}
\end{minipage} \\
\hline
Semana 3 & \begin{minipage}{.85\textwidth}
\begin{itemize} \itemsep-0.4em
	\vspace{1mm}
	\item \textbf{Manejo de datos} en R (``\textit{data wrangling}").
	\vspace{1mm}
\end{itemize}
\end{minipage} \\
\hline
Semana 4 & \begin{minipage}{.85\textwidth}
\begin{itemize} \itemsep-0.4em
	\vspace{1mm}
	\item \textbf{Introducción a \textit{ggplot2}}\\
	Estructura de la \textit{gramática de los gráficos} (aplicado a \verb=R=).
	\vspace{1mm}
\end{itemize}
\end{minipage} \\
\hline
Semana 5 & \begin{minipage}{.85\textwidth}
\begin{itemize} \itemsep-0.4em
	\vspace{1mm}
	\item \textbf{Reportes, presentaciones y documentos con \texttt{RMarkdown}}
	\vspace{1mm}
\end{itemize}
\end{minipage} \\
\hline
Semana 6 & \begin{minipage}{.85\textwidth}
\begin{itemize} \itemsep-0.4em
	\vspace{1mm}
	\item \textbf{Un paso más allá} con \textit{ggplot} para geolocalizar información.
	\vspace{1mm}
\end{itemize}
\end{minipage} \\
\hline\hline
Semana 7 & \begin{minipage}{.85\textwidth}
\begin{itemize} \itemsep-0.4em
	\vspace{1mm}
	\item \textbf{Estadística} descriptiva e inferencial para el análisis político.
	\vspace{1mm}
\end{itemize}
\end{minipage} \\
\hline
Semana 8 & \begin{minipage}{.85\textwidth}
\begin{itemize} \itemsep-0.4em
	\vspace{1mm}
	\item \textbf{Análisis de datos} en R:
	 la regresión de \textit{Mínimos Cuadrados Ordinarios} (OLS) como puerta de entrada al mundo de las inferencias.
	\vspace{1mm}
\end{itemize}
\end{minipage} \\
\hline
Semana 9 & \begin{minipage}{.85\textwidth}
\begin{itemize} \itemsep-0.4em
	\vspace{1mm}
	\item \textbf{Análisis de datos} en R (II): \textit{Modelos lineales generalizados} (GLM).
	\vspace{1mm}
\end{itemize}
\end{minipage} \\
\hline \hline

Semana 10 & \begin{minipage}{.85\textwidth}
\begin{itemize} \itemsep-0.4em
	\vspace{1mm}
	\item \textbf{Introducción a tópicos intermedios y avanzados}.
		
	\vspace{1mm}
\end{itemize}
\end{minipage} \\
\hline
\hline

Semana 11 & \begin{minipage}{.85\textwidth}
\begin{itemize} \itemsep-0.4em
	\vspace{1mm}
	\item \textbf{Introducción a tópicos intermedios y avanzados (II)}.
		
	\vspace{1mm}
\end{itemize}
\end{minipage} \\
\hline
\hline
\end{tabular} 
\end{table}

\clearpage

\subsection*{Semana 1}
        
\textbf{(A)} Descripción de los 3 grandes módulos del taller: (1) Manejo  y análisis de datos; (2) presentación de resultados y (3) producción de documentos científicos. \textbf{(B)} Tareas preparatorias para el taller: programas necesarios + encuesta. 
        
\subsection*{Semana 2}

¿Por qué \verb=R=? Funcionamiento general. Primeros pasos con \verb=RStudio= (consola, \textit{scripts}, \textit{enviroment}, ayuda, paquetes, etc.) 

    \begin{itemize}
        \item \textbf{Práctica}: procesamiento de la encuesta a participantes (estadísticas descriptivas y gráficos con  \verb=R base=).
    \end{itemize}
  
\subsection*{Semana 3}
        Presentación de \verb=tidyverse= para el manejo y limpieza de bases de datos (\textit{data wrangling}): 
        
    \begin{itemize}
            \item Aplicación de \verb=readr= para importar bases de datos, \verb=tidyr= para limpiarlos y \verb=dplyr= para transformarlos.
    \end{itemize}   
    
\subsection*{Semana 4}

    \begin{itemize}
        \item ¿Qué es la 
        \textit{Gramática de los Gráficos}?
        Leland Wilkinson y una nueva manera de visualizar la información estadística.
        
        \item ¿Cómo se aplica a \verb=R= mediante \verb=ggplot2=? 
        \item Los 5 fantásticos de \verb=ggplot=: ejemplos prácticos para la creación de \textit{scatterplots, linegraphs, boxplots, histograms y barplots.}
        \item \textbf{Ejemplo}: \textit{``El impacto de enchufar los votos"}, (Ruiz Nicolini, 2017).
    \end{itemize}

\subsection*{Semana 5}

Uso de \verb=RMarkdown= para la producción de documentos científicos (artículos, informes y presentaciones) y su relación con \LaTeX: 

    \begin{itemize}
        \item Descripción de las herramientas y versiones disponibles.
        \item Comandos básicos para la redacción de un documento.
        \item Más allá del texto: fórmulas matemáticas, manejo de figuras, tablas, citas, bibliografía, referencias e índices en la preparación de documentos científicos.
        \item Presentaciones usando \verb=beamer= y \verb=RMarkdown=.
    \end{itemize}
    
\subsection*{Semana 6}
Representación de información con mapas, grillas y polígonos. Ejemplos aplicados con:
        \begin{itemize}
            \item \verb=ggmap=
            \item \verb=geofacet= 
            \item \verb=sf=
        \end{itemize}

\subsection*{Semana 7}

Los métodos de estadística descriptiva e inferencial    
    \begin{itemize}
        \item Medidas de centralidad y dispersión.
        \item Técnicas exploratorias para el análisis: la reducción de dimensiones.
        \item Tests de hipótesis: diferencia de medias e independencia de variables.
        \item Coeficientes de correlación y asociaciones entre variables.
    \end{itemize}   

\subsection*{Semana 8}
 Análisis estadístico en Ciencia Política. 
    \begin{itemize}
            \item La regresión lineal como primera aproximación a la detección de regularidades.
            \item Anscombe y los riesgos de la regresión lineal.
            \item Los supuestos y su verificación.
            \item El problema de la multicolinealidad.
            \item Variantes robustas a los Mínimos Cuadrados Ordinarios.
            \item \textbf{Estadísticas descriptivas}. Ejercicio práctico: análisis de base de datos de encuesta a participantes de prueba piloto de voto electrónico en la provincia de Salta (Monogan, 2015).
            \item \textbf{Análisis de regresión}. Estimando la democratización de los países mediante sus fronteras. 
    \end{itemize}

\subsection*{Semana 9}
Primera aproximación a \textit{Modelos lineales generalizados} (GLM). 
        
    \begin{itemize}
        \item ¿Qué es lo que se generaliza en un modelo \textit{generalizado}?
        \item Los distintos sabores de regresiones: logística, Ridge y LASSO.
        \item Regularización y variaciones en los atributos a predecir.
        \item  \textbf{Práctica}: \textit{¿cómo predecir si una persona va a votar en las elecciones?}
    \end{itemize}     
    
\subsection*{Semanas 10 y 11}

Introducción a tópicos intermedios y avanzados de análisis estadístico:
    \begin{itemize}
        \item Regresiones multinivel con \verb=lme4=.
        \item Regresiones bayesianas con \verb=rstanarm=.
        \item Análisis de series de tiempo en modelos de regresión lineal.
    \end{itemize}
\end{document}



